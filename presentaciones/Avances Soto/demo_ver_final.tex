\documentclass[10pt]{article}
\usepackage[utf8]{inputenc}
\usepackage[activeacute,spanish,es-nodecimaldot]{babel}
\usepackage[left=1.5cm,top=1.5cm,right=1.5cm, bottom=1.5cm,letterpaper, includeheadfoot]{geometry}
%\usepackage[parfill]{parskip}

\usepackage{amssymb, amsmath, amsthm}
\usepackage{graphicx}
\usepackage{lmodern,url}
\usepackage{paralist} %util para listas compactas

% paquetes agregados por mi
% para escribir algoritmos
\usepackage{algorithm}
\usepackage[noend]{algpseudocode}


\usepackage{tikz}
%\usetikzlibrary{positioning,chains,fit,shapes,calc}
\usetikzlibrary{positioning, shapes, arrows,decorations.text}
\usepackage{float}
\usepackage{caption}
\usepackage{ mathdots }
\usepackage{verbatim}
\usepackage{ stmaryrd }
\usepackage{fancyhdr}
\pagestyle{fancy}
\fancypagestyle{plain}{%
\fancyhf{}
\lhead{\footnotesize\itshape\bfseries\rightmark}
\rhead{\footnotesize\itshape\bfseries\leftmark}
}


% macros
\newcommand{\Q}{\mathbb Q}
\newcommand{\R}{\mathbb R}
\newcommand{\N}{\mathbb N}
\newcommand{\Z}{\mathbb Z}
\newcommand{\C}{\mathbb C}

% macros jp
\newcommand{\cev}[1]{\reflectbox{\ensuremath{\vec{\reflectbox{\ensuremath{#1}}}}}}
\newcommand{\sfunction}[1]{\textsf{\textsc{#1}}}
\algrenewcommand\algorithmicforall{\textbf{foreach}}
\algrenewcommand\algorithmicindent{.8em}
\newcommand{\val}[1]{\text{val}(#1)}

%Teoremas, Lemas, etc.
\theoremstyle{plain}
\newtheorem{teo}{Teorema}
\newtheorem{lem}{Lema}
\newtheorem{prop}{Proposición}
\newtheorem{cor}{Corolario}

\theoremstyle{definition}
\newtheorem{defi}{Definición}
% fin macros

%%%%% NOMBRE ESCRIBAS Y FECHA
\newcommand{\sca}{Escriba Uno}
\newcommand{\scb}{Escriba Dos}
\newcommand{\scc}{Escriba Tres}
\newcommand{\catnum}{0} %numero de catedra
\newcommand{\fecha}{5 de noviembre 2018 }

%%%%%%%%%%%%%%%%%%

%Macros para este documento
\newcommand{\cin}{\operatorname{cint}}

\begin{document}
%Encabezado
\fancyhead[L]{Facultad de Ciencias Físicas y Matemáticas}
\fancyhead[R]{Universidad de Chile}
\vspace*{-1.2 cm}
%\begin{minipage}{0.6\textwidth}
%\begin{flushleft}
%\hspace*{-0.5cm}\textbf{MA4702. Programación Lineal Mixta. 2018.}\\
%\hspace*{-0.5cm}\textbf{Profesor:} José Soto\\
%\hspace*{-0.5cm}\textbf{Escriba(s):} \sca, \scb~y \scc.\\
%\hspace*{-0.5cm}\textbf{Fecha:} \fecha.
%\end{flushleft}
%\end{minipage}
\begin{flushleft}
\includegraphics[scale=0.15]{fcfm}
\end{flushleft}
\bigskip
%Fin encabezado


\subsection{Algunos resultados de aproximación para ATMP (caso ganancias lineales)}

Sea $P$ el problema ATMP (asginaci\'on de trabajos en m\'aquinas paralelas). Se estudiar\'an algoritmos de aproximaci\'on para el caso en que las curvas de ganancia para cada trabajo $i \in [n]$ tienen la forma:


%\begin{center}
%\textit{Se tienen $n$ trabajos y $m$ m\'aquinas. Cada trabajo $i \in [n]$ tiene asociado una tasa de ganancia $p_i$ y un tiempo $x_i$ tales que la ganancia $g_i(t)$ obtenida por procesar dicho trabajo $t$ unidades de tiempo est\'a dada por}
$$
g_i(t) = \left\{\begin{array}{lr}
p_i t & t \leq t_i \\
p_i t_i & t>t_i
\end{array}\right.
$$
donde $p_i$ es la tasa de ganancia asociada al trabajo $i-$\'esimo, y $t_i$ es el tiempo de proceso m\'aximo despu\'es del cual el trabajo $i$ deja de aportar beneficio. Suponagmos que los trabajos est\'an indexados de modo que:
\begin{equation}
\label{eq:trabajosOrd}
p_1 > p_2 > \cdots > p_n
\end{equation}

%\textit{El objetivo es encontrar una asignaci\'on de trabajos a m\'aquinas de m\'aximo beneficio tal que:}
%\begin{enumerate}
%\item Sea $t_j$ el instante en que termina el \'ultimo trabajo la m\'aquina $j-$\'esima. Entonces $t_j \leq 1$.
%
%\item 
%\end{enumerate}
%\end{center}



%Sea $\val{\text{OPT}}$ el valor de alguna asignación \'optima para el problema $P$. A continuación se enuncian un par de proposiciones que serán \'utiles m\'as adelante (la demostración del segundo queda pendiente).

\begin{defi}[Asignación]
Una asignación $\mathcal{A}$ es una tupla $(A,T,M,X)$ donde:
\begin{itemize}
\item $A \subseteq [n]$ es el conjunto de \textit{trabajos asignados}.
\item $T = \{y_i\}_{i \in A}$ son los \textit{tiempos de proceso} de cada trabajo $i \in A$.
\item $M = \{m_i\}_{i \in A}$ es el conjunto de máquinas asignadas para cada trabajo en $A$.
%\item $X$ es el conjunto de \textit{posiciones} de procesamiento de cada trabajo $i \in A$ en $m_i$, i.e. $i$ es el $x_i-$ésimo trabajo procesado en la máquina $m_i$.
\end{itemize}
\end{defi}

Se denotará por $\val{\mathcal{A}}$ al \textit{valor} de la asignación $\mathcal{A}$, y diremos que un trabajo fue \textit{cortado} según la asignación $\mathcal{A}$ si $0<y_i < t_i$. 
\begin{lem}
\label{lem:cortes}
Sea $\mathcal{A}$ una asignación factible. Entonces se puede obtener una asignación $\mathcal{A}'$ a partir de $\mathcal{A}$ de modo que solo se haya cortado un trabajo por máquina, y $\val{\mathcal{A}'} \geq \val{\mathcal{A}}$.
\end{lem}
\begin{proof}
Sean $i,j$ dos trabajos cortados según $\mathcal{A}$, sin pérdida de generalidad supongase que  $p_i \geq p_j$. El tiempo disponible que le queda al trabajo $i$ para poder ser procesado y aumentar su contribución al beneficio está dado por $t_i-y_i$. Luego, basta hacer los cambios $y_i' \gets y_i + \min\{t_i-y_i,y_j\}$, $y_j' \gets y_j -\min\{t_i-y_i,y_j\}$, y definir $T'$ con estos nuevos tiempos de proceso $y_i'$ y $y_j'$ para los trabajos $i$ y $j$, respectivamente. Definiendo $\mathcal{A}'$ con estos nuevos tiempos de proceso, se tiene que, o bien $i$ se realiza en $t_i$, o bien el trabajo $j$ se deja de realizar. Luego el numero de trabajos cortados disminuye al menos en $1$. Por otro lado, dado que $p_i \geq p_j$ se tiene que:
$$
\val{\mathcal{A}'} \geq \val{\mathcal{A}}
$$
\end{proof}

Otro resultado de utilidad es el siguiente.
\begin{lem}
\label{lem:caracOPT}
Sea $\mathcal{A} = (A,T,M)$ alguna asignación óptima para el problema $P$. Se tiene entonces que: 
\begin{equation}
\label{eq:caractOPT}
p_i \geq p_j \quad \forall i \in A,\ j \in [n]\setminus{A}
\end{equation}
i.e., el óptimo asigna los trabajos con mejor pendiente.
\end{lem}
\begin{proof}
Por contradicción, supongase que existen $i\in A,j \in [n]\setminus{A}$ tales que no se cumple (\ref{eq:caractOPT}). Entonces haciendo el cambio $y_i \gets 0$ y $y_j \gets \min\{y_i,t_j\}$, se tendría un asignamiento $\mathcal{A}'$ tal que:
$$
\val{\mathcal{A}'} > \val{\mathcal{A}}
$$
lo cual contradide la optimalidad de  $\mathcal{A}$.
\end{proof}
De los lemas (\ref{lem:cortes}) y (\ref{lem:caracOPT}) se concluye directamente el siguiente resultado:
\begin{prop}
\label{prop:final}
Existe una asignación $\mathcal{A} = (A,T,M)$ óptima para $P$ que cumple las siguientes características:
\begin{enumerate}
\item  $p_i \geq p_j$, para $i \in A$, $j \in [n]\setminus{A}$.
\item Toda máquina corta a lo más un trabajo.
\end{enumerate}
\end{prop}

\textbf{Obs:} De la proposición (\ref{prop:final}) y de el orden de las pendientes en (\ref{eq:trabajosOrd}) se puede deducir que existe una asignación óptima tal que el conjunto de trabajos asignados tiene la forma $A = [\ell]$ para algún $\ell \in [n]$.\\~\\

%\begin{proof}
%En efecto, considere una máquina $j$ y supongamos que existen dos trabajos $j_1$ y $j_2$ tales que cada uno se procesa un tiempo $h_1 < x_{j_1}$ y $h_2 < x_{j_2}$, respectivamente. Sin perdida de generalidad, supongamos que $p_1 \geq p_2$. Llamese por $A$ la presente asignaci\'on. Se separar\'a el an\'alisis en diferentes casos:
%\begin{itemize}
%\item \textbf{caso $h_2 \geq x_{j_1}-h_1$}:
%
%En este caso, como $p_1 \geq p_2$, basta con reemplazar el tiempo de proceso de $j_2$ por tiempo de proceso de $j_1$, de modo que $j_1$ no sea cortado, y $j_2$
%\end{itemize}
%\end{proof}

Para encontrar un algoritmo de aproximaci\'on, dividiremos el problema según el \textit{largo} de los trabajos. Para ello, sea $k \in \N$ fijo.
\subsubsection*{caso $x_i \leq 1/k \quad \forall i \in [n]$}

%Sea $P$ el problema ATMP, y considerese $P'$ el problema donde, en vez de tener la restricción de dead line igual a $1$, ahora solo se pide que cada trabajo asignado haya partido antes de $1$. Además, considere que los largos de cada trabajo cumplen la relación $x_i \leq 1/k$, para algún $k \in \N$ fijo. Como antes, hay $n$ trabajos a agendar y $m$ el número de máquinas. Pir último, considere el siguiente algoritmo:

%Para el problema $P$, sea $\text{OPT}$ el valor de la asignación optima. A continuación enunciamos un par de proposisiones que serán utiles:

Para el presente caso se mostrar\'a que un algoritmo glot\'on logra una $(1+1/k)^2$ aproximaci\'on. Considere para ello una variante del problema $P$, llamada $P'$, donde se realizan los siguientes cambios en las restricciones: 
\begin{enumerate}
\item Todo asignación $\mathcal{A}$ tiene que conjunto de trabajos asignados de la forma $A=[\ell]$, para algún $\ell \in [n]$. 
\item Para todo $i \in A$, $y_i = t_i$.
\item Todo trabajo agendado debe partir antes de $1$. 
\end{enumerate}

Para describir el algoritmo Considere las siguientes notaciones. Sea $j \in [m]$ una m\'aquina cualquiera:
\begin{itemize}
\item $L_j$: instante en que la máquina $j$ termina el \'ultimo trabajo.
%\item $t$: instante m\'as chico en que se termina el \'ultimo trabajo asignado.
\item $\ell$: \'ultimo trabajo asignado.
\item $m_\ell$: m\'aquina a la que se asigna el trabajo $\ell$ (un valor de $0$ indica que no ha sido asignado).
\item $x_\ell$: instante en el que el trabajo $\ell$ parte.
\end{itemize}
Con esto se puede describir un algoritmo glot\'on para $P'$:
\begin{algorithm}[H]
\caption{Glotón para $P'$}\label{alg:greddyP'}
\begin{algorithmic}[1]

\State ordenar las pendientes de modo que $p_1 \geq p_2 \geq p_3 \geq \cdots \geq p_n$  \;
\State $\ell \gets 0$\;
\For{$i: 1 \to n$}
	\State $L_i \gets 0$\;
	\State $m_i \gets 0$\;
\EndFor
\While {$\min\{L_j:j \in [m]\}\leq 1$}
	\State $\ell \gets \ell+1$\;
	\State $j \gets \text{argmin}\{L_j: j \in [m]\}$\;
	\State $m_\ell \gets j$\;
	\State $L_j \gets \sum_{i = 1}^n \llbracket m_i=m_\ell\rrbracket t_i + t_\ell$
	\State $x_\ell \gets L_j-t_\ell$\;
\EndWhile
\State \textbf{end}
\State \textbf{Return } $(A = [\ell], X = \{x_i\}_{i=1}^\ell, M = \{m_i\}_{i=1}^{\ell})$
\end{algorithmic}
\end{algorithm}


Se denotará por ALG al algoritmo (\ref{alg:greddyP'}). Antes de probar que ALG induce una $(1+1/k)^2$ aproximación para el problema $P$, un resultado que es de utilidad:

\begin{prop}
\label{prop:appP'}
ALG es una $1+1/k$ aproximación para el problema $P'$.
\end{prop}
\begin{proof}
En efecto, sea $\mathcal{A}_{P'}$ una asignación óptima para $P'$, y sea $\mathcal{A}_{\text{ALG}}$ la asignación entregada por ALG, y considere $A_{P'} = [\ell_{P'}]$ y $A_{\text{ALG}}= [\ell_{\text{ALG}}]$. Dada la condición (\ref{eq:trabajosOrd}), sabemos el último trabajo asignado por ALG cumple que
\begin{equation}
\label{eq:minorante}
p_{\ell_{\text{ALG}}} \leq p_i \quad \forall i \in [\ell_{\text{ALG}}]
\end{equation}
Por otro lado, es claro que $\ell_{\text{ALG}} \leq \ell_{\text{OPT}}$. Denotemos por $\val{\text{ALG}}$ y $\val{\text{OPT}}$ los valores de la función objetivo para los agendamientos de ALG y OPT respectivamente. Entonces:
\begin{align*}
\val{\text{OPT}} &= \sum_{i = 1}^{\ell_{\text{OPT}}} p_i t_i \\
		  &= \sum_{i=1}^{\ell_\text{ALG}}p_i t_i + \sum_{i = \ell_{\text{ALG}}+1}^{\ell_{\text{OPT}}}p_it_i\\
		  &\leq \val{\text{ALG}}+p_{\ell_\text{ALG}} \sum_{i = \ell_{\text{ALG}}+1}^{\ell_{\text{OPT}}}t_i \\
		  &\leq \val{\text{ALG}}+p_{\ell_\text{ALG}} + p_{\ell_\text{ALG}} m\frac{1}{k} \qquad (\star)
\end{align*}
donde $(\star)$ por otro lado, notemos que:
\begin{align*}
\val{\text{ALG}} &= \sum_{i=1}^{\ell_{\text{ALG}}} p_i t_i \\
		  &\geq p_{\ell_{\text{ALG}}}\sum_{i=1}^{\ell_{\text{ALG}}} t_i\\
		  &\geq p_{\ell_{\text{ALG}}}m
\end{align*}
Así, juntando los dos desarrollos anteriores se concluye que:
$$
\val{\text{OPT}} \leq \left(1 + \frac{1}{k}\right) \val{\text{ALG}}
$$
\end{proof}
De la proposcición (\ref{prop:appP'}) se deduce el siguiente corolario:
\begin{cor}
ALG induce una asignación para el problema $P$ que es una $(1+1/k)^2$ aproximación.
\end{cor}
\begin{proof}
Sean $\val{\text{OPT}^P}$ y $\val{\text{OPT}^{P'}}$ los valores óptimos de $P$ y $P'$, respectivamente. Más aún, consideremos la asignación óptima de $P$ con la estructura descrita en la observación de la proposición (\ref{prop:final}). Dicha asignación para $P$ tiene beneficio dado por $\val{\text{OPT}^P}$, y ademas cumple con ser una asignación factible para $P'$, luego:
\begin{equation}
\label{eq:PP'}
\val{\text{OPT}^P} \leq \val{\text{OPT}^{P'}}
\end{equation}

gracias a la proposición (\ref{prop:appP'}), se tiene además que:
$$
\val{\text{OPT}^P} \leq \val{\text{OPT}^{P'}} \leq \left(1+\frac{1}{k} \right)\val{\text{ALG}}
$$

Por otro lado, el beneficio $\val{\text{ALG}}$ puede ser dividido en el valor obtenido por los trabajos procesados hasta el instante $1$, y el valor asociado a los que se procesaron en instantes posteriores
\begin{equation}
\label{eq:divP'}
\val{\text{ALG}} = \val{\text{ALG}_{>1}} + \val{\text{ALG}_{\leq 1}}
\end{equation}

Se demostrar\'a entonces que
\begin{equation}
\label{eq:cotaP'}
\val{\text{ALG}_{>1}} \leq \frac{1}{k}\val{\text{ALG}_{\leq 1}}
\end{equation}

Notar que de (\ref{eq:cotaP'}), (\ref{eq:divP'}) y (\ref{eq:PP'}) se deduce directamente que:
\begin{equation*}
\val{\text{OPT}^P} \leq \left( 1 + \frac{1}{k} \right)^2\val{\text{ALG}_{\leq 1}}
\end{equation*}

Así, basta probar (\ref{eq:cotaP'}) para concluir el resultado. Para ello, considere por un momento los trabajos asignados a una máquina $j \in [m]$, y abusando de notación llamemos $p_1 \ldots p_l$ las pendientes de los trabajos asignados a dicha máquina, realizados en el mismo orden de indexación. El tiempo que ALG procesa el último trabajo en la máquina $j$, $t_\ell$, puede ser escrito como $t_\ell = \tilde{t}_\ell + t'_\ell$, donde $\tilde{t}_\ell$ es la cantidad de tiempo que ALG procesa $\ell$ hasta el instante $1$, y $t'_\ell$ es el tiempo que se procesa $\ell$ después del instante $1$. Sea $\val{\text{ALG}_{j,\leq 1}}$ el valor asociado a los trabajos que se procesan hasta el instante $1$ en la máquina $j$, i.e.
\begin{equation}
\label{eq:ganjP}
\val{\text{ALG}_{j,\leq 1}}  = \sum_{i <\ell} p_i t_i + p_\ell \tilde{t}_\ell
\end{equation} 
Luego, notemos que $1 = \sum_{i<\ell} t_i + \tilde{t}_{\ell}$, de donde

\begin{equation*}
 p_\ell(1-\tilde{t}_\ell) = p_\ell\sum_{i < \ell} t_i = \sum_{i < \ell} p_\ell t_i \leq  \sum_{i < \ell} p_i t_i
\end{equation*}
donde la última desigualdad viene del hecho de que ALG asigna los trabajos en orden decreciente de pendiente, de modo que $p_\ell \leq p_i \ \forall i \in [\ell]$. Del último desarrollo se tiene que
\begin{equation}
\label{eq:cotasGan}
p_\ell \leq \sum_{j < \ell} p_j t_j + p_\ell \tilde{t}_\ell
\end{equation}
usando que $t_\ell \leq 1/k$, se deduce que
\begin{equation}
\label{eq:cotaAntesDps}
p_\ell t'_\ell \leq \frac{1}{k}\left(\sum_{j < \ell} p_j t_j + p_\ell \tilde{t}_\ell \right) = \frac{1}{k}\val{\text{ALG}_{j,\leq 1}}
\end{equation}
Así, considere ahora $\ell_1, \ldots, \ell_m$ los \'ultimos trabajos que cada m\'aquina procesa seg\'un ALG, y sus respectivos $\{p_{\ell_i}\}_{i=1}^m$ y $\{t_{\ell_i}\}_{i=1}^m$. Notar que
$$
\val{\text{ALG}_{>1}} = \sum_{i = 1}^m p_{\ell_i}t'_{\ell_i}, \quad   \val{\text{OPT}_{\leq 1}}  = \sum_{i=1}^m \text{ALG}^{P'}_{j_i,\leq 1}  
$$
Entonces, por la desigualdad (\ref{eq:cotaAntesDps}) se puede deducir que
\begin{equation*}
\val{\text{ALG}_{>1}} = \sum_{i = 1}^m p_{\ell_i} t'_{\ell_i} \leq \sum_{i=1}^m \frac{1}{k} \val{\text{ALG}_{j_i,\leq 1}} = \frac{1}{k} \val{\text{ALG}_{\leq 1}}
\end{equation*}
lo que termina la demostración.
\end{proof}
\newpage
Se puede lograr una $(1+1/k)$ aproximación del problema $P$ sin pasar por $P'$, usando el siguiente hecho:
\begin{lem}
\label{lem:cotaell}
Sean $\mathcal{A}_P = (A_P = [\ell_P], T^P =\{y^P_i\}_{i=1}^{\ell_P},M^P)$ y $\mathcal{A}_{\text{ALG}} = (A_{\text{ALG}} = [\ell_{\text{ALG}}], T^{\text{ALG}} =\{y^{\text{ALG}}_i\}_{i=1}^{\ell_{\text{ALG}}},M^{\text{ALG}})$, las asignaciones del óptimo para P y la de ALG, respectivamente, tomando a $\mathcal{A}_P$ con la estructura dada por la proposición (\ref{prop:final}). Entonces
\begin{equation*}
\ell_P \leq \ell_{\text{ALG}}
\end{equation*}
\end{lem}
\begin{proof}
Por contradicción, supongamos que $\ell_P > \ell_{\text{ALG}}$. Con lo anterior, sabemos que $P$ debe procesar a los trabajos $1\ldots \ell_{\text{ALG}}$, y por otro lado sabemos que 
$$
\sum_{i = 1}^{\ell_{\text{ALG}}} \llbracket m^{\text{ALG}}_i = j\rrbracket y^{\text{ALG}}_i = 1 \qquad \forall j \in [m] 
$$  
Es decir, sabemos que los tiempos de proceso para los trabajos $1\ldots \ell_{\text{ALG}}$ dados por $T^{\text{ALG}}$ son tales que utilizan todo el tiempo disponible por maquina. Lo anterior implica que debe existir algún trabajo $i \in [\ell_{\text{ALG}}]$ tal que $t_i \geq y^{\text{ALG}}_i > y^P_i$, y $m^P_i = m^P_{\ell_P+1}$ (es decir, $i$ y $\ell_P + 1$ son asignados a la misma m\'aquina por $\mathcal{A}_P$). Entonces, se puede mejorar la asignaci\'on realizada por el \'optimo, asignando m\'as tiempo de proceso a $i$ y correspondientemente asignandole menos tiempo de proceso a $\ell_P + 1$. Para ello, sea $t>0$ definido por

$$
t := \min\{t_i-y^P_i,y^P_{\ell_P + 1}\}
$$
Así, llamemos $\mathcal{A}'_P$ a la asignaci\'on que se obtiene de modificar $\mathcal{A}_P$ al reemplazar $y^P_i \gets y^P_i+t$, $y^P_{\ell_P +1} \gets y^P_{\ell_P +1}-t$, se tiene que la asignación resultante es factible y tiene valor estrictamente más grande que $\mathcal{A}_P$ (pues $p_i > p_{\ell_P + 1}$), contradiciendo la optimalidad de $\mathcal{A}_P$. 
\end{proof}

Con lo anterior, se puede obtener el siguiente resultado:
\begin{cor}
\label{cor:appPchico}
ALG induce una asignaci\'on que es una $(1+1/k)$ aproximación para el problema P.
\end{cor}

\begin{proof}
En efecto, consideremos la notación del lema (\ref{lem:cotaell}). Como $A_P \subseteq A_{\ell_{\text{ALG}}}$, y dado que $t_i \geq y^P_i$ para $i \in [\ell_{\text{ALG}}]$, se tiene que

\begin{align*}
\val{\mathcal{A}_\text{P}} &= \sum_{i=1}^{\ell_P} p_i y^P_i \\
						&\leq \sum_{i=1}^{\ell_P} p_i t_i \\
						&\leq \sum_{i=1}^{\ell_{\text{ALG}}} p_i t_i = \val{\mathcal{A}_\text{ALG}}\\
\end{align*}
Luego, de igual manera que en demostraciones anteriores, se puede demostrar que:

\begin{align*}
\val{\mathcal{A}_\text{ALG}} &= \val{\mathcal{A}_\text{ALG}}_{\leq 1} + \val{\mathcal{A}_\text{ALG}}_{> 1}\\
\val{\mathcal{A}_\text{ALG}}_{> 1} &\leq \frac{1}{k} \val{\mathcal{A}_\text{ALG}}_{\leq 1}
\end{align*}

donde $\val{\text{ALG}}_{\leq 1}$ es el valor obtenido por los trabajos procesados hasta el instante $1$, mientras que $\val{\text{ALG}}_{>1}$ es valor de los trabajos procesados despu\'es del instante $1$. Con lo anterior se concluye que:
$$
\val{\mathcal{A}_\text{P}} \leq \left(1 + \frac{1}{k} \right)\val{\mathcal{A}_\text{ALG}}_{\leq 1}
$$ 
lo que termina la demostraci\'on
\end{proof}

Consideremos por \'ultimo una variaci\'on del problema $P$, que llamaremos $P_C$, donde $C$ es un conjunto de capacidades para cada m\'aquina, esto es $C = \{c_j\}_{j = 1}^m$, con $c_j \geq c$ para alg\'un $c>0$, para todo $j\in [m]$, y en vez de suponer que $t_i \leq 1/k$ para cada trabajo, suponer que $kt_i \leq c$ para $i \in [n]$. Luego, se tiene el siguiente resultado:
\begin{lem}
\label{lem:CasoCapVar}
Con los supuestos de arriba, se tiene que ALG induce una $(1+1/k)$ aproximaci\'on. 
\end{lem}
\begin{proof}
An\'alogamente a lo que se ha demostrado en los resultados anteriores, se tiene que
$$
\val{\mathcal{A}_P} \leq \val{\mathcal{A}_{\text{ALG}}}
$$
Lo anterior viene del hecho de que para esa cota nunca fue relevante el tama\~no de los trabajos, si no que $\ell_P \leq \ell_{\text{ALG}}$. La \'unica parte donde se utilizo que $t_i \leq 1/k$ fue para mostrar que:
$$
\val{\mathcal{A}_{\text{ALG}}}_{> 1} \leq \frac{1}{k} \val{\mathcal{A}_{\text{ALG}}}_{ \leq 1}
$$
Para mostrar lo anterior, consideremos la $j$-\'esima m\'aquina, denotando  por $\val{\mathcal{A}_{\text{ALG}}}_{j}$ el valor que aportan los trabajos asignados a la $j$-ésima máquina. Para simplificar notación, denotemos por $\{p^j_k\}_{k=1}^{\ell_j}$ a las tasas de ganancia de los trabajos asignados a la máquina $j$, donde $\ell_j$ es el último trabajo asignado, y por $\{y^j_k\}_{k=1}^{\ell_j}$ a los tiempos de proceso de dichos trabajos según glotón. El tiempo de proceso del último trabajo puede dividirse en dos sumandos, i.e.

$$
y^j_{\ell_j} = y^j_{\ell_j, \leq 1} + y^j_{\ell_j, > 1}
$$

Es decir, el tiempo de proceso del último trabajo  puede dividirse en el tiempo que se utilizó antes del instante $1$ ($y^j_{\ell_j, \leq 1}$) y el tiempo de proceso después del instante $1$.  Por otro lado, tenemos que
$$
y^j_{\ell_j, > 1} = \sum_{k=1}^{\ell_j} y^j_k-c_j
$$
y como $k y^j_{\ell_j, > 1}\leq k t_{\ell_j}\leq c \leq c_j$, sigue que
$$
y^j_{\ell_j, > 1} \leq \sum_{k=1}^{\ell_j}y^j_k -ky^j_{\ell_j, > 1}
$$
Entonces, usando que $p^j_{\ell_j}<p^j_k$ para $k \in [\ell_j - 1]$, se puede deducir que
\begin{align*}
\val{\mathcal{A}_{\text{ALG}}}_{j,>1} &= p^j_{\ell_j}y^j_{\ell_j, > 1}\\
									  &\leq \sum_{k=1}^{\ell_j} y^j_kp^j_{\ell_j} - kp^j_{\ell_j}y^j_{\ell_j, > 1}\\
									  &= \sum_{k=1}^{\ell_j-1} y^j_kp^j_{\ell_j} + y^j_{\ell_j}p^j_{\ell_j}-kp^j_{\ell_j}y^j_{\ell_j, >1}\\
									  &\leq \sum_{k=1}^{\ell_j-1}y^j_k p_k^j + y^j_{\ell_j}p^j_{\ell_j}-kp^j_{\ell_j}y^j_{\ell_j, >1}\\
									  &=\val{\mathcal{A}_{\text{ALG}}}_{j,\leq 1} + y^j_{\ell_j,>1}p^j_{\ell_j}-kp^j_{\ell_j}y^j_{\ell_j,>1}\\
									  &=\val{\mathcal{A}_{\text{ALG}}}_{j,\leq 1}-(k-1)\underbrace{p^j_{\ell_j}y^j_{\ell_j,>1}}_{\val{\mathcal{A}}_{j,>1}}\\
\end{align*}
Luego
$$
k \val{\mathcal{A}_{\text{ALG}}}_{j,>1} \leq \val{\mathcal{A}_{\text{ALG}}}_{j,\leq 1}
$$

Así, sumando sobre las máquinas la desigualdad anterior, se puede deducir que 
$$
k \val{\mathcal{A}}_{>1} \leq \val{\mathcal{A}}_{\leq 1}
$$
y entonce se concluye la demostración notando que
$$
\val{\mathcal{A}_P} \leq \val{\mathcal{A}_{\text{ALG}}} = \left(1+ \frac{1}{k}\right)\val{\mathcal{A}_{\text{ALG}}}_{\leq 1}
$$
\end{proof} 
Con el lema anterior se tiene que, para que el algoritmo (\ref{alg:greddyP'}) induzca una $(1+1/k)$ aproximación para  el problema $P$, es suficiente que el tiempo disponible por máquina para procesar trabajos sea al menos $k$ veces el largo de los trabajos (en verdad, $k$ veces una cota superior para el largo de los trabajos).

Con esto en mente, ahora se procederá a estudiar el caso en que no todos los trabajos cumplen $t_i \leq 1/k$, y hay trabajos \textit{largos} (i.e. $t_i > 1/k$). Para ello, primero supongamos que disponemos de la asignación de los trabajos largos en cada máquina (i.e.):

\begin{enumerate}
\item Los trabajos largos asignados que se procesan completamente (i.e. en los tiempos $t_i$).
\item De ser el caso, el trabajo largo cortado en dicha máquina.
\item El tiempo disponible por máquina, i.e. el tiempo que no fue utilizado por los trabajos largos.
\item Por último, y esto en general, se dispone de $\ell$, los trabajos que serán asignados. 
\end{enumerate} 

Eventualmente, podría pasar que hay máquinas sin trabajos largos asignados. Como  
%Recordemos que algoritmo (\ref{alg:greddyP'}) retorna $\mathcal{A}_{\text{ALG}} = ((A = [\ell_{\text{ALG}}], X = \{x_i\}_{i=1}^{\ell_{\text{ALG}}}, M = \{m_i\}_{i=1}^{\ell_{\text{ALG}}}))$. Para simplif
\newpage
\subsubsection*{Caso $t_i \leq 1/k$, para $k \in \N$}

Para obtener una aproximación del valor óptimo para el problema $P$, considérese la siguiente variante de éste: 
\begin{enumerate}
\item Todo asignación $\mathcal{A}$ tiene que conjunto de trabajos asignados de la forma $A=[\ell]$, para algún $\ell \in [n]$. 
\item Para todo $i \in A$, $y_i = t_i$.
\item Todo trabajo asignado debe partir antes de $1+1/k$. 
\end{enumerate}

Para el presente caso, se utiliza el mismo algoritmo glotón (\ref{alg:greddyP'}), pero cambiando la restricción de \texttt{while} menor que $1$ por $1+1/k$. Luego se tiene el siguiente resultado.

\begin{lem}
\label{lem:eles}
Sean $\mathcal{A}_{\text{ALG}} = (A_{\text{ALG}} = [\ell_{\text{ALG}}], T_{\text{ALG}}, M_{\text{ALG}})$ y $\mathcal{A}_{P} = (A_P = [\ell_P], T_P, M_P)$ las asignaciones de ALG y de algún óptimo para $P$, con la forma de la asignación dada por la proposición (\ref{prop:final}). Entonces $\ell_P \leq \ell_{\text{ALG}}$.
\end{lem}

\begin{proof}

\end{proof}

Consideremos las asignaciones enunciadas en el lema anterior, y denotemos por $\val{\mathcal{A}_{\text{ALG}}}$ y $\val{\mathcal{A}_P}$ los valores respectivos. Por el lema anterior, y dado que los trabajos asignados por ALG se procesan de manera completa (i.e. en los $t_i$'s), se tiene que
$$
\val{\mathcal{A}_P} = \sum_{i = 1}^{\ell_P} p_i y^P_i \leq \sum_{i = 1}^{\ell_{\text{ALG}}} y^P_i p_i \leq \sum_{i = 1}^{\ell_{\text{ALG}}}t_i p_i=\sum_{i = 1}^{\ell_{\text{ALG}}} y^{\text{ALG}}_i p_i=\val{\mathcal{A}_{\text{ALG}}}
$$

Esto es

\begin{equation}
\label{eq:PauxMayorP}
\val{\mathcal{A}_P} \leq \val{\mathcal{A}_{\text{ALG}}}
\end{equation}

Por otro lado, se tiene el siguiente lema:

\begin{lem}
\label{lem:cotaExtra}
Sea $\val{\mathcal{A}_{\text{ALG}}}$ el valor de la asignación para $P'$. Digamos que $\val{\mathcal{A}_{\text{ALG}}}=\val{\mathcal{A}_{\text{ALG}}}_{\leq 1} + \val{\mathcal{A}_{\text{ALG}}}_{>1}$, i.e. $\val{\mathcal{A}_{\text{ALG}}}$ se puede dividir en el valor por los trabajos que se procesaron antes del instante $1$, y el valor correspondiente a los que se procesaron en instantes posteriores a $1$. Entonces:
\begin{equation}
\label{eq:cotaExtra}
\val{\mathcal{A}_{\text{ALG}}}_{> 1} \leq \frac{2}{k} \val{\mathcal{A}_{\text{ALG}}}_{\leq 1}
\end{equation}
\end{lem}

\begin{proof}
Se puede probar igual que las anteriores, notando los trabajos asignados después del instante $1$ no serán procesados  una cantidad de tiempo superior a $2/k$. 
\end{proof}

\begin{cor}
ALG induce una $1+2/k$ aproximación para el problema P, dada por los trabajos procesados hasta el instante 1.
\end{cor}
\begin{proof}

\end{proof}

Antes de terminar esta sección, consideremos una variante del problema $P$ donde se sabe que al menos hay $1-\gamma>0$ de tiempo disponible en cada máquina, y que cada trabajo $i$ cumple que $t_i \leq \alpha$, para algún $\alpha >0$. Entonces, de manera análoga a las demostraciones anteriores, se tiene que:

\begin{lem}
\label{lem:capCalpha}
Con las condiciones enunciadas arriba, $\val{\mathcal{A}_{\text{ALG}}}_{>1} \leq \displaystyle \frac{1-\gamma}{2\alpha}\val{\mathcal{A}_{\text{ALG}}}_{\leq 1}$
\end{lem}

\begin{cor}
\label{cor:cAlpha}
ALG induce una $1+(1-\gamma)/2\alpha$ aproximación para el problema $P$.
\end{cor}
\newpage
\subsubsection*{Caso $m$ constante.}

Para este caso, consideremos que en vez de $k \in \N$, tenemos un parámetro $\alpha>0$ como en la demostración del lema (\ref{lem:capCalpha}), de modo que los trabajos $i$ tales que $t_i \leq \alpha$ son los trabajos cortos, y en caso contrario se llamarán trabajos largos.\\~\\

Supongamos por un momento que somos capaces de adivinar la las asignaciones de los trabajos grandes en el óptimo, i.e. se sabe qué trabajos son asignados por $\mathcal{A}_P$ a cada máquina, de modo que también somos capaces de saber cuanto tiempo libre hay para los trabajos cortos. Sea $\gamma \in (0,1)$, y supongamos que en cada máquina se pondera el tiempo de proceso asignado a los trabajos largos por $\gamma$. Como los tiempos de proceso, por máquina, de los trabajos largos no pueden sumar más que $1$, se tiene que al menos hay $1- \gamma$ de espacio disponible para los trabajos pequeños por máquina. Dado lo anterior, tendremos que:
$$
\val{\mathcal{A}_\text{ALGext}} = \val{\mathcal{A}_{\text{ALGext}}}_{L} + \val{\mathcal{A}_{\text{ALGext}}}_{c} \geq  \gamma \val{\mathcal{A}_P}_{L} + \frac{1}{1+ \frac{1-\gamma}{2 \alpha}}\val{\mathcal{A_P}}_{c}
$$
Estudiemos si hay algún $\gamma$ tal que
$$
\gamma  = \frac{1}{1+\frac{1-\gamma}{2\alpha}}
$$

Esta última ecuación equivale a
$$
\gamma^2-(2\alpha+1)\gamma -2\alpha = 0
$$
La cual tiene solución
$$
\gamma_\pm = \frac{2\alpha+1\pm\sqrt{(2\alpha+1)^2-8\alpha}}{2} = \frac{2\alpha+1\pm|2\alpha-1|}{2}
$$

Luego, desarrollando se llega a que hay dos posibles soluciones a priori:

$$
\gamma_1 = 2 \alpha \qquad \gamma_2 = 1
$$

Notamos que $\gamma_2$ no es una solución factible cuando se considera $\gamma \in (0,1)$, y entonces la única opción es $\alpha < 1/2$ y usar $\gamma_1$, así se obtiene que
$$
\val{\mathcal{A}_{\text{ALGext}}} \geq 2\alpha \val{\mathcal{A}_P}
$$

Lo cual es un PTAS cuando el número de máquinas es constante, pues basta tomar $\varepsilon >0$ y $\alpha:= 1/2-\varepsilon/2$, obteniendo:

$$
\val{\mathcal{A}_{\text{ALG}}} \geq (1-\varepsilon)\val{A_P}
$$
%Para este caso, se puede resolver el problema de manera exacta en tiempo polinomial si se consideran $m$ y $k$ constantes. Para ello basta notar que la cantidad de trabajos que puede tener cada m\'aquina es menor que $km$, y entonces la cantidad de asignaciones factibles para $P$ en este caso es $O(m^{km})$

\subsubsection*{Caso $m$ variable.}

Para el caso de $m$ variable
\end{document}