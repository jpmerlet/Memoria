\documentclass[10pt]{article}
\usepackage[utf8]{inputenc}
\usepackage[activeacute,spanish,es-nodecimaldot]{babel}
\usepackage[left=1.5cm,top=1.5cm,right=1.5cm, bottom=1.5cm,letterpaper, includeheadfoot]{geometry}
\usepackage[parfill]{parskip}

\usepackage{amssymb, amsmath, amsthm}
\usepackage{graphicx}
\usepackage{lmodern,url}
\usepackage{paralist} %util para listas compactas

% paquetes agregados por mi
\usepackage[linesnumbered,lined,commentsnumbered, ruled]{algorithm2e}
\usepackage{tikz}
%\usetikzlibrary{positioning,chains,fit,shapes,calc}
\usetikzlibrary{positioning, shapes, arrows,decorations.text}
\usepackage{float}
\usepackage{caption}
\usepackage{ mathdots }
\usepackage{verbatim}
\usepackage{fancyhdr}
\pagestyle{fancy}
\fancypagestyle{plain}{%
\fancyhf{}
\lhead{\footnotesize\itshape\bfseries\rightmark}
\rhead{\footnotesize\itshape\bfseries\leftmark}
}


% macros
\newcommand{\Q}{\mathbb Q}
\newcommand{\R}{\mathbb R}
\newcommand{\N}{\mathbb N}
\newcommand{\Z}{\mathbb Z}
\newcommand{\C}{\mathbb C}

% macros jp
\newcommand{\cev}[1]{\reflectbox{\ensuremath{\vec{\reflectbox{\ensuremath{#1}}}}}}

%Teoremas, Lemas, etc.
\theoremstyle{plain}
\newtheorem{teo}{Teorema}
\newtheorem{lem}{Lema}
\newtheorem{prop}{Proposición}
\newtheorem{cor}{Corolario}

\theoremstyle{definition}
\newtheorem{defi}{Definición}
% fin macros

%%%%% NOMBRE ESCRIBAS Y FECHA
\newcommand{\sca}{Escriba Uno}
\newcommand{\scb}{Escriba Dos}
\newcommand{\scc}{Escriba Tres}
\newcommand{\catnum}{0} %numero de catedra
\newcommand{\fecha}{5 de noviembre 2018 }

%%%%%%%%%%%%%%%%%%

%Macros para este documento
\newcommand{\cin}{\operatorname{cint}}

\begin{document}
%Encabezado
\fancyhead[L]{Facultad de Ciencias Físicas y Matemáticas}
\fancyhead[R]{Universidad de Chile}
\vspace*{-1.2 cm}
%\begin{minipage}{0.6\textwidth}
%\begin{flushleft}
%\hspace*{-0.5cm}\textbf{MA4702. Programación Lineal Mixta. 2018.}\\
%\hspace*{-0.5cm}\textbf{Profesor:} José Soto\\
%\hspace*{-0.5cm}\textbf{Escriba(s):} \sca, \scb~y \scc.\\
%\hspace*{-0.5cm}\textbf{Fecha:} \fecha.
%\end{flushleft}
%\end{minipage}
\begin{flushleft}
\includegraphics[scale=0.15]{fcfm}
\end{flushleft}
\bigskip
%Fin encabezado

%-----------------------------------------
%--------Aquí comienza el documento-------
%-----------------------------------------



\begin{center}
\LARGE\textbf{Avances 2}\\
\fecha
\end{center}
\bigskip

Consideremos el problema de agendamiento de trabajos en máquinas idénticas, donde cada trabajo tiene su curva de ganancia $f^j$ asociada. Recordemos que $P_1$ es el problema de agendamiento donde no se permiten interrupciones en la ejecución de los trabajos (es decir, una vez que se comienza un trabajo $j$ en un día, en los días siguientes se elije continuar dicho trabajo, o comenzar otro y desechar $j$ como un trabajo disponible para los siguientes días). Llamamos $P_2$ es el mismo problema, pero ahora permitiendo interrupciones en la ejecución de un trabajo en una máquina, pero sin poder realizar un trabajo en diferentes máquinas aún. Por último, $P_3$ es el trabajo donde se permite interrupción y realizar un mismo trabajo en diferentes máquinas. Para que el problema tenga sentido, se considerará que $n > m$\\~\\

Para este último problema, consideremos que para cada trabajo $j$ se puede descomponer cada función de ganancia como:
\begin{equation}
\label{eq:desc}
f^j(t) = g^j(1) + g^j(2) + \cdots + g^j(i),\ \forall i \in [T]
\end{equation}
donde supondremos que $g^j(i)\geq g^j(i+1)$. Notar que en lo anterior queda explicito el supuesto de que cada trabajo se puede trabajar durante $T$ días. Antes de escribir el glotón que resuelve $P_3$, llamemos $G\in \mathcal{M}_{[n] \times [T]}$ a la \textit{matriz de ganancia}, de modo que $G_{ji}$ = $g^j(i)$. A cada entrada de $G_{ji}$ le llamaremos \textit{subtrabajo}. Escribiremos el agendamiento resultante de glotón en una secuencia de multi-conjuntos $\{S_t\}_{t \in [T]}$, donde cada multi-conjunto tendrá los trabajos que se realizarán el día $i$-ésimo. Dicho esto, el siguiente pseudo-código explica lo que realiza glotón: 



\begin{algorithm}

\caption{Glotón para $P_3$}
\label{alg:gloton}
\SetKwData{Left}{left}\SetKwData{This}{this}\SetKwData{Up}{up}
\SetKwFunction{Union}{Union}\SetKwFunction{FindCompress}{FindCompress}
\SetKwInOut{Input}{input}\SetKwInOut{Output}{output}


\Input{$G$, matriz de ganancia}
\Output{$S_i$, para $i \in [T]$}

\BlankLine


$T_1$ $\leftarrow \{G_{j1}:j\in [n]\}$ \emph{inicializar $T_d$, multi-conjunto de trabajos disponibles}\;
\For{$i\leftarrow 1$ \KwTo $T$}{
%\emph{special treatment of the first element of line $i$}\;
%\For{$j\leftarrow 2$ \KwTo $w$}{\label{forins}
$S_{i}$ $\leftarrow$ los $m$ mejores subtrabajos de $T_t$\;
$T_{i+1}$ $\leftarrow$ $T_t\setminus{S_i} \cup \{G_{j,k+1}: \text{para $k\in [T]$, $j \in [n]$, tal que $G_{jk}\in S_{i}$}\}$\;
}
\Return $S_i$, para $i \in [T]$
%}
%\If(\tcp*[f]{O(\Up,\This)==1}){\Up compatible with \This}{\label{ut}
%\lIf{\Up $<$ \This}{\Union{\Up,\This}}
%\tcp{\This is put under \Up to keep tree as flat as possible}\label{cmt}
%\lElse{\Union{\This,\Up}}\tcp*[h]{\This linked to \Up}\label{lelse}
%}
%}
%\lForEach{element $e$ of the line $i$}{\FindCompress{p}}
%}
%\caption{disjoint decomposition}\label{algo_disjdecomp}
\end{algorithm}
Para verificar que el resultado del algoritmo (\ref{alg:gloton}) es el optimo del problema $P_3$, se mostrará que el resultado es óptimo en el problema equivalente de flujo de costo mínimo en una red a especificar.\\~\\


Antes de definir la red, recordemos que ésta consiste en una tupla $(\vec{D}, u, c)$, donde $\vec{D} = (V, \vec{E})$ es el digrafo donde se quiere buscar el flujo máximo; $u: \vec{E} \to \N$ una función que a cada arco asocia la \textit{capacidad} máxima de flujo que se le envía; y $c:\vec{E} \to \R_-^*$ es la función que a cada arco le asocia un \textit{costo}. Un flujo $f:\vec{E} \to \R_+$ es una función que a cada arco de $\vec{D}$ asocia un real no negativo a enviar por ese arco.
Dado un flujo $f$ para una red $(\vec{D}, u, c)$, para determinar si dicho flujo es óptimo usaremos el concepto de \textit{red residual} $(\vec{D}^f, u^f,c^f)$. Dicho digrafo se construte como sigue: 
\begin{enumerate}
\item Definir $\vec{D}' = (V, \vec{E} \cup \cev{E})$, es decir, al digrafo $\vec{D}$ le agregamos los arcos en el sentido contrario.

\item  Para las capacidades $u^f$ residuales: a cada arco $(i,j) \in \vec{E}$ le asociamos $u^f_{ij} = u_{ij}-f_{ij}$, y para $(j,i) \in \cev{E}$, $u^f_{ji} = f_{ij}$.  Para los costos tenemos que $c^f_{ij} = c_{ij}$ y $c^f_{ji} = -c_{ij}$. Sea $\vec{E}^f$ los arcos tales que su capacidad residual es positiva.

\item Por último, el digrafo residual está dado por $\vec{D}^f = (V, \vec{E}^f)$, y entonces la \textit{red residual} será $(\vec{D}^f, u^f, c^f)$. 
\end{enumerate}  


\tikzset{%
  every neuron/.style={
    circle,
    draw,
    minimum size=.6cm
  },
  neuron missing/.style={
    draw=none, 
    scale=2,
    text height=0.333cm,
    execute at begin node=\color{black}$\vdots$
  },
  arc rdiag/.style={
    draw=none, 
    scale=2,
    text height=0.333cm,
    execute at begin node=\color{black}$\iddots$
  },
  arc fdiag/.style={
    draw=none, 
    scale=2,
    text height=0.333cm,
    execute at begin node=\color{black}$\ddots$
  },
  arc vert/.style={
    draw=none, 
    scale=2,
    text height=0.333cm,
    execute at begin node=\color{black}$\vdots$
  },
}


\begin{figure}[H]
\begin{center}
\begin{tikzpicture}[x=1.5cm, y=1.5cm]%, >=stealth]

% nodo sumidero t
\foreach \m [count=\y] in {t}
  \node [every neuron/.try, neuron \m/.try, label=center:$\m$] (t) at (-2,0-\y) {};

% nodo fuente
\foreach \m [count=\y] in {s}
  \node [every neuron/.try, neuron \m/.try, label=center:$\m$] (s) at (4,0-\y) {};

% nodos de la capa de dias
\foreach \m/\l [count=\y] in {1,2,missing,3}
  \node [every neuron/.try, neuron \m/.try] (input-\m) at (0,1.5-\y) {};
  
% nodos de la capa de trabajos
\foreach \m [count=\y] in {1,2,missing,3}
  \node [every neuron/.try, neuron \m/.try] (hidden-\m) at (2,2-\y*1.25) {};

% labels nodos trabajos
\foreach \l [count=\i] in {1,2,n}
  \node [above, label= below:$j_\l$] at (hidden-\i.north) {};
  
% labels nodos días
\foreach \l [count=\i] in {1,2,T}
  \node [above, label= below:$\l$] at (input-\i.north) {};

% arcos entre fuente y capa de trabajos
\foreach \i in {1,2,3}
    \draw [<-] (hidden-\i) to [bend left]  (s);

% arcos entre fuente y capa de trabajos (arcos superiores)
\def\myshift#1{\raisebox{1ex}}
\foreach \i in {1,2,3}
    %\draw [<-,postaction={decorate,decoration={text along path,text align=center,text={|\myshift| {$(0,g^\i(1))$}{}}}}] (hidden-\i) to [bend left]   (s);
    \draw [<-] (hidden-\i) to [bend left]  (s);

% arcos entre fuente y capa de trabajos (arcos inferiores)
\foreach \i in {1,2,3}
    \draw [<-] (hidden-\i) to [bend right]  (s);

% arcos entre la capa de trabajos y días
\foreach \i in {1,2,3}
  \foreach \j in {1,2,3}
    \draw [<-] (input-\i) -- (hidden-\j);

% arcos entre la capa de días y sumidero
\foreach \i in {1,2,3}
    \draw [<-] (t) -- (input-\i);

% dots para los arcos faltantes entre fuente y trabajo
\node [arc rdiag/.try] (arc1) at (3, -0.1) {};
\node [arc vert/.try] (arc2) at (3, -0.7) {};
\node [arc fdiag/.try] (arc3) at (3, -2) {};
\foreach \l [count=\x from 0] in {Days, Works}
  \node [align=center, above] at (\x*2,1.5) {\l};

\end{tikzpicture}

\end{center}
\caption{Digrafo $\vec{D}$ de la red para resolver el problema $P_3$.}
\label{fig:digRed}
\end{figure}

En la Figura (\ref{fig:digRed}) se puede el digrafo de la red que se utilizará para desmotrar la optimalidad del algoritmo (\ref{alg:gloton}). Para especificar cuales son los costos y las capacidades asociados a cada arco, ver (\ref{fig:sourceWork}), (\ref{fig:workDays}), y (\ref{fig:daysSumidero}).\\~\\

\tikzset{%
  arc rdiag/.style={
    draw=none, 
    scale=1.5,
    text height=0.333cm,
    execute at begin node=\color{black}$\iddots$
  },
}

\captionsetup{justification=centering,margin=2cm}
\begin{figure}[H]
\captionsetup{justification=centering,margin=2cm}
\begin{center}
\begin{tikzpicture}
\begin{scope}[every node/.style={circle,draw}]
    \node (trabajo) at (0,3) {$j_i$};
    \node (source) at (2.5,1) {$s$};
\end{scope}
\def\myshift#1{\raisebox{1ex}}
\begin{scope}[every label/.style={fill=white,circle}]
	\draw [<-,postaction={decorate,decoration={text along path,text align=center,text={|\myshift| {$(-g^i(1),1)$}}}}] (trabajo) to [bend left]   (source);
\def\myshift#1{\raisebox{-2.5ex}}
	\draw [<-,postaction={decorate,decoration={text along path,text align=center,text={|\myshift| {$(-g^i(T),1)$}}}}] (trabajo) to [bend right]  (source);
\end{scope}
\node [arc rdiag/.try] (arc1) at (1.35, 2) {};
\tikzset{every label/.style={fill=white,circle}}

\end{tikzpicture}
\end{center}
\caption{Cada arco tiene una etiqueta $(-g^i(k),1)$, donde la primera coordenada es el costo y la segunda la capacidad de dicho arco.}
\label{fig:sourceWork}
\end{figure}

\begin{figure}[H]
\begin{center}
\begin{tikzpicture}[x=1.5cm, y=1.5cm, scale = .77]%,>=stealth]                                      

% nodo trabajo
\foreach \m [count=\y] in {trabajo}
  \node [every neuron/.try, neuron \m/.try, label=center:$j_i$] (trabajo) at (2,0-\y) {};

% nodos de la capa de dias
\foreach \m/\l [count=\y] in {1,2,missing,3}
  \node [every neuron/.try, neuron \m/.try] (input-\m) at (0,1.5-\y) {};
  
% labels nodos días
\foreach \l [count=\i] in {1,2,T}
  \node [above, label= below:$\l$] at (input-\i.north) {};
\def\myshift#1{\raisebox{1ex}}
% arcos entre la capa de trabajos y días
\foreach \j in {1,2,3}
	\draw [<-, postaction={decorate,decoration={text along path,text align=center,text={|\myshift| {$(0,1)$}}}}] (input-\j) -- (trabajo);

\foreach \l [count=\x from 0] in {Days, Works}
  \node [align=center, above] at (\x*2,1) {\l};

\end{tikzpicture}
\end{center}
\caption{A diferencia de los arcos que van desde la fuente a la capa de trabajos, aquí todos los costos son iguales a cero.}
\label{fig:workDays}
\end{figure}

\begin{figure}[H]
\begin{center}
\begin{tikzpicture}[x=1.5cm, y=1.5cm, scale = 0.9]%,>=stealth]                                      

% nodo trabajo
\foreach \m [count=\y] in {trabajo}
  \node [every neuron/.try, neuron \m/.try, label=center:$t$] (trabajo) at (0,0-\y) {};

% nodos de la capa de dias
\foreach \m/\l [count=\y] in {1,2,missing,3}
  \node [every neuron/.try, neuron \m/.try] (input-\m) at (2,1.5-\y) {};
  
% labels nodos días
\foreach \l [count=\i] in {1,2,T}
  \node [above, label= below:$\l$] at (input-\i.north) {};
\def\myshift#1{\raisebox{1ex}}
% arcos entre la capa de trabajos y días
\foreach \j in {1,2,3}
	\draw [<-, postaction={decorate,decoration={text along path,text align=center,text={|\myshift| {$(0,1)$}}}}] (trabajo) -- (input-\j);

\foreach \l [count=\x from 0] in {, Days}
  \node [align=center, above] at (\x*2,1) {\l};

\end{tikzpicture}
\end{center}
\caption{Cada arco tiene una etiqueta $(-g^i(k),1)$, donde la primera coordenada es el costo y la segunda la capacidad de dicho arco.}
\label{fig:daysSumidero}
\end{figure}

Sea $f^G$ el flujo dado por glotón, tal que $f^G(\vec{E}) \subset \{0,1\}$. Esto úlitmo viene del hecho de que el algoritmo (\ref{alg:gloton}) asigna $m$ subtrabajos a cada día. Luego una asignación se interpreta en terminos del flujo equivalente a que el flujo valga $1$ si el subtrabajo fue asignado, y cero en caso contrario. \\~\\

El teorema (\ref{teo:certificado}) nos da un certificado para la optimalidad del flujo $f^G$:\\~\\
\begin{teo}
\label{teo:certificado}
Sea $(\vec{D}, u, c)$ una red dada, y $f$ un flujo factible. $f$ es un flujo de costo mínimo si y sólo si la red residual $(\vec{D}^f, u^f, c^f)$ no contiene ciclos dirigidos de costo negativo. 
\end{teo}

Luego, veamos que $(\vec{D}^{f^G}, u^{f^G}, c^{f^G})$ no tiene ciclos de costo negativo. Para ello, primero notemos que, de haber ciclos de costo negativo, no hay arcos entre las capas Days y $t$ que pudan participar de dichos ciclos. Esto último pues, como para cada día el algoritmo (\ref{alg:gloton}) asigna $m$ subtrabajos, entonces la capacidad residual de los arcos $it$ con $i \in [T]$ es $u^{f^G}(i,t) = 1-f^G_{it} = 0$, luego ninguno de los arcos $it$ aparecen en la red residual. Al mismo tiempo, los únicos arcos que aparecen entre estas dos capas son los arcos reversos (ver Figura (\ref{fig:daysSumideroResid}))

\begin{figure}[H]
\begin{center}
\begin{tikzpicture}[x=1.5cm, y=1.5cm, scale = 0.9]%,>=stealth]                                      

% nodo trabajo
\foreach \m [count=\y] in {trabajo}
  \node [every neuron/.try, neuron \m/.try, label=center:$t$] (trabajo) at (0,0-\y) {};

% nodos de la capa de dias
\foreach \m/\l [count=\y] in {1,2,missing,3}
  \node [every neuron/.try, neuron \m/.try] (input-\m) at (2,1.5-\y) {};
  
% labels nodos días
\foreach \l [count=\i] in {1,2,T}
  \node [above, label= below:$\l$] at (input-\i.north) {};
\def\myshift#1{\raisebox{1ex}}
% arcos entre la capa de trabajos y días
\foreach \j in {1,2,3}
	\draw [<-, postaction={decorate,decoration={text along path,text align=center, reverse path,text={|\myshift| {$(0,1)$}}}}] (input-\j) -- (trabajo);
% nombres de las capas
\foreach \l [count=\x from 0] in {, Days}
  \node [align=center, above] at (\x*2,1) {\l};

\end{tikzpicture}
\end{center}
\caption{Arcos entre el sumidero $t$ y la capa Days en el grafo residual.} 
\label{fig:daysSumideroResid}
\end{figure}
Por otro lado, lo que pasa entre las capas Days y Works no es tan simple como lo anterior, pues sabemos que cada $i \in [T]$ debe recibir $m$ arcos con flujo $1$, pero dichos arcos dependerán de los trabajos disponibles para dicho día. Sin embargo, se es fácil notar que solo hay $3$ tipos de ciclos que podemos encontrar en el digrafo residual $\vec{D}^{f^G}$:

\begin{enumerate}
\item  Ciclos entre el nodo fuente $s$ y un trabajos dado $j_i$. Notar que esto último ocurre pues entre el nodo fuente y cualquier trabajo tenemos $T$ arcos. Sea $d_i \in [T]$ la cantidad de días que glotón asigno a $j_i$. Los ciclos son como en la Figura ().
  \begin{figure}[H]
\captionsetup{justification=centering,margin=2cm}
\begin{center}
\begin{tikzpicture}
\begin{scope}[every node/.style={circle,draw}]
    \node (trabajo) at (0,3) {$j_i$};
    \node (source) at (2.5,1) {$s$};
\end{scope}
\def\myshift#1{\raisebox{1ex}}
\begin{scope}[every label/.style={fill=white,circle}]
	\draw [<-,postaction={decorate,decoration={text along path,text align=center,text={|\myshift| {$(-g^i(1),1)$}}}}] (trabajo) to [bend left]   (source);
\def\myshift#1{\raisebox{-2.5ex}}
	\draw [<-,postaction={decorate,decoration={text along path,text align=center,text={|\myshift| {$(-g^i(T),1)$}}}}] (trabajo) to [bend right]  (source);
\end{scope}
\node [arc rdiag/.try] (arc1) at (1.35, 2) {};
\tikzset{every label/.style={fill=white,circle}}

\end{tikzpicture}
\end{center}
\caption{Cada arco tiene una etiqueta $(-g^i(k),1)$, donde la primera coordenada es el costo y la segunda la capacidad de dicho arco.}
\label{fig:sourceWorkResid}
\end{figure} 
\end{enumerate}

\newpage
\section{Maximización de trabajos en máquinas idénticas (MTMI), caso sin intercambio}

Lo primero que se presentará es que el problema de $2$- partición se puede reducir a MTMP. Para fijar ideas, consideremos que cualquier instancia de MTMI se representará mediante la tupla $(\{g^i\}_{i=1}^n,\{x_i\}_{i=1}^n, T, m)$, donde $\{g^i\}_{i=1}^n$ corresponde a las curvas de ganancias marginales para cada trabajo $i \in [n]$; $\{x_i\}_{i=1}^n \subset \N$ son tales que $g^i(t) = 0$ para todo $t>x_i$; T el dead line, y finalmente $m$ al número de máquinas.\\~\\

\subsection{Reducción de $2$-partición a MTMI.}
\begin{defi}(2-partición)
\label{def:2part}
Sea $S \subset \N$ un conjunto de naturales, tales que $\sum_{x \in S} x = 2 K$, para $K \in \N$. Dados $S$ y $K$, el problema de decisión asociado a 2 partición consiste en saber si existe una partición de $S = S_1 \cup S_2$ tal que:
$$
\sum S_1 = \sum S_2 = K
$$
\end{defi}

Para la reducción, sean entonces $S, K$ como en la definición (\ref{def:2part}), más precisamente sea $S = {x_1, \ldots, x_n}$. Para cada $i \in [n]$ consideremos:
\begin{equation}
\label{eq:red}
g^i(t) = 1 \quad \forall t \in [x_i]
\end{equation}
Dadas las ganancias marginales anteriores para los trabajos $i \in [n]$, consideremos un dead line $T:= K$ y $m = 2$. Así, a partir de una instancia $(S, K)$ para el problema de $2$-partición se define una instancia $(\{g^i\}_{i=1}^n,\{x_i\}_{i=1}^n, T, m)$. Luego, veamos que:

\begin{equation}
\label{eq:red2part}
(S, K)\ \text{admite $2$-partición} \Longleftrightarrow (\{g^i\}_{i=1}^n,\{x_i\}_{i=1}^n, K, 2)\  \text{admite óptimo de valor } 2K
\end{equation}


\subsection{Reducción de $3$-partición a MTMI.}
\begin{defi}{(3-partición)}
Sea $S\subset \N$ tal que $|S| = 3m$, para algún $m \in \N$. $3$-partición corresponde al problema de desición siguiente: Es posible particionar el conjunto $S$ en $m$ $3$-tuplas, de manear que cada tupla sume lo mismo?
\end{defi}

Dicho lo anterior, veamos que $3$-partición se reduce a MTMI. Consideremos un tipo particular de instancia (para el cual $3$-partición sigue siendo un problema fuertemente NP-completo): Para fijar ideas, digamos que $\sum S = m K$, para algún $K \in \N$. Supondremos entonces que nuestras instancias cumplen que:

\begin{equation}
\label{eq:inst}
K/4 < x < K/2\  \text{para } x \in S
\end{equation}



Así, una instancia de $3$-partición la codificaremos como una tupla $(S, K, m)$. Análogo al caso anterior, definimos $3m$ trabajos $\{g^i\}_{i=1}^{3m}$ de igual manera que en (\ref{eq:red}). Consideremos dead line $T:=K$, y consideremos $m$ máquinas. Se tiene que:

\begin{equation}
\label{eq:red3part}
(S, K, m) \text{ admite $3$-partición} \Longleftrightarrow (\{g^i\}_{i=1}^{3m}, \{x_i\}_{i=1}^{3m}, T, m)
\text{ admite óptimo de valor $Km$}
\end{equation}

Para la equivalencia (\ref{eq:red3part}), es clave la condición  (\ref{eq:inst}), pues así se asegura que la única forma de que el óptimo para MTMI sea $Km$ es que cada máquina logre utilizar exactamente 3 trabajos (pues si alguna máquina queda con tiempo libre, entonces dicha asignación de trabajos no es óptima). La condición (\ref{eq:inst}) implica que, de tener MTMI valor óptimo igual a $Km$, la única forma es que cada máquina este llena con tres trabajos cada una, puese si alguna tuviese dos entonces ahí su valor es menor que $K$, y ya no podrá tenerse la 3-partición. Pues, dadas las características de la instancia MTMI (\ref{eq:red3part})

Cada vez que se puede tener 3-particion, es claro (también para $2$-partición!) que se tendrá valor óptimo para MTMI igual a $Km$ (respectivamente $2K$).
\end{document}
